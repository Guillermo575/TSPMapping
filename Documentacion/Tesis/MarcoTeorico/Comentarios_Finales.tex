\subsection {Comentarios finales}
Como se acaba de ver la resolución de problemas es un tema bastante extenso, aunque teóricamente se puede resolver mediante el método de fuerza bruta que consiste en probar con todas las combinaciones existentes y tomar la correcta, es el más tardado y en algunos casos imposibles de resolver, por eso se busca otros métodos capaces de encontrar una buena solución, no necesariamente tiene que ser la mejor, pero si lo más cercano o bien, que pueda resolver una parte especifica del problema.\\
%\hspace*{1cm}El tema es tan complejo que se tiene una clasificación de complejidad de acuerdo a la longitud y variedad de parámetros de cada problema, algunos como el TSP crecen de manera exponencial.\\
\hspace*{1cm}Uno de los algoritmos mencionados fue el de agrupamiento (Clustering) que permite dividir todo el conjunto de parámetros en diferentes grupos y de esa manera simplificar el proceso sin tener que realizar un proceso con soluciones de más y que a simple vista son ineficientes.\\
\hspace*{1cm}Además de los clásicos algoritmos que son capaces de dar la misma solución cada vez que se usa la respectiva fórmula, existen también las metaheurísticas que permite obtener secuencias aleatorias que proporciona soluciones que en los algoritmos convencionales no se permitiría ver, aunque son soluciones generadas al azar están controlados mediante ciertas normas que hacen que no se salgan del camino pero puedan seguir buscando diferentes soluciones.\\
\hspace*{1cm}Como comentario final, se pueden combinar los 2 tipos de algoritmos, primero usando algoritmos determinístas para crear una solución inicial, y luego aplicar metaheurísticas para ir explorando zonas alrededor de la solución inicial y poder encontrar una mejor solución.\\ 