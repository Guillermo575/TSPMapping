\subsection {Comentarios finales}
Como últimos comentarios cabe señalar que una de las ventajas del método de cuadrantes es que, al tratarse de un método determinista, permite obtener un resultado de manera rápida y acertada sin depender de elementos aleatorios. Aunque la desventaja más notable es que siempre se obtendrá la misma respuesta y no siempre será la mejor posible, aquí es donde entraron las metaheurísticas que permitieron obtener mejores resultados sin tener que revisar manualmente cada desperfecto de la solución anterior.\\
\hspace*{1cm}La razón por la que se comentó acerca del arte con TSP fue para demostrar la enorme utilidad que tiene el método de cuadrantes en otro tipo de aplicación real aparte de los problemas cotidianos de TSP, además de que fueron los problemas que más puntos abarcaron, el de la Mona Lisa de 100 mil puntos y la de Vermeer con 200 mil.\\
%\hspace*{1cm}Por ultimo haberlas comparado con los resultados obtenidos en otras instituciones hace pensar que todavía falta mucho por perfeccionar para no depender del uso de las metaheurísticas, aunque es más que satisfactorio que los resultados obtenido mediante estos experimentos no hayan superado más allá del 20\% de rendimiento.
\clearpage \newpage