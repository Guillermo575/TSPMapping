\subsection{Resumen de experimentos}
Después de presentar los experimentos se recolectaron y vaciaron los resultados en las siguientes tablas cuyas descripciones se detallarán más adelante: 

\begin{itemize}
    \item Experimentos aplicando el método de cuadrantes (tabla \ref{table:ExperimentosConCuadrantes}).
    \item Experimentos sin aplicar el método de cuadrantes (tabla \ref{table:ExperimentosSinCuadrantes}).
    \item Análisis global de los mejores resultados (tabla \ref{table:ExperimentosComparativos}).
\end{itemize}
        
\subsubsection{Experimentos aplicando el método de cuadrantes}  
 La tabla \ref{table:ExperimentosConCuadrantes} muestra los resultados obtenidos de los experimentos aplicando el método de cuadrantes y aplicando una de las 3 metaheurísticas en cada uno de ellos. De las 100 corridas que se realizó sobre cada problema se tomó la mejor solución, la peor y se hizo un promedio de todos los resultados obtenidos.
 
% Please add the following required packages to your document preamble:
% \usepackage[table,xcdraw]{xcolor}
% If you use beamer only pass "xcolor=table" option, i.e. \documentclass[xcolor=table]{beamer}
\begin{table}[hbtp]
\centering
\caption{Experimentos aplicando el método de cuadrantes.}
\resizebox{1\textwidth}{!}{
\rotatebox{0}{

\begin{tabular}{lrrrrrrrrrr}
\multicolumn{2}{c}{}                                                                                                                                                     & \multicolumn{9}{c}{\cellcolor[HTML]{343434}{\color[HTML]{FFFFFF} \textbf{Experimentos aplicando método de cuadrantes}}}                                                                                                                                                                                                                                                                                                                                                                                                                                                                                                                                                                                                                                                                                              \\
\multicolumn{2}{l}{}                                                                                                                                                     & \multicolumn{3}{c}{\cellcolor[HTML]{656565}{\color[HTML]{FFFFFF} \textbf{Recocido Simulado}}}                                                                                                                                                               & \multicolumn{3}{c}{\cellcolor[HTML]{656565}{\color[HTML]{FFFFFF} \textbf{Greedy}}}                                                                                                                                                                          & \multicolumn{3}{c}{\cellcolor[HTML]{656565}{\color[HTML]{FFFFFF} \textbf{Genético}}}                                                                                                                                                                        \\
\rowcolor[HTML]{9B9B9B} 
\multicolumn{1}{c}{\cellcolor[HTML]{9B9B9B}{\color[HTML]{FFFFFF} \textbf{Nombre}}} & \multicolumn{1}{c}{\cellcolor[HTML]{9B9B9B}{\color[HTML]{FFFFFF} \textbf{Ciudades}}} & \multicolumn{1}{c}{\cellcolor[HTML]{9B9B9B}{\color[HTML]{FFFFFF} \textbf{Peor}}} & \multicolumn{1}{c}{\cellcolor[HTML]{9B9B9B}{\color[HTML]{FFFFFF} \textbf{Mejor}}} & \multicolumn{1}{c}{\cellcolor[HTML]{9B9B9B}{\color[HTML]{FFFFFF} \textbf{Promedio}}} & \multicolumn{1}{c}{\cellcolor[HTML]{9B9B9B}{\color[HTML]{FFFFFF} \textbf{Peor}}} & \multicolumn{1}{c}{\cellcolor[HTML]{9B9B9B}{\color[HTML]{FFFFFF} \textbf{Mejor}}} & \multicolumn{1}{c}{\cellcolor[HTML]{9B9B9B}{\color[HTML]{FFFFFF} \textbf{Promedio}}} & \multicolumn{1}{c}{\cellcolor[HTML]{9B9B9B}{\color[HTML]{FFFFFF} \textbf{Peor}}} & \multicolumn{1}{c}{\cellcolor[HTML]{9B9B9B}{\color[HTML]{FFFFFF} \textbf{Mejor}}} & \multicolumn{1}{c}{\cellcolor[HTML]{9B9B9B}{\color[HTML]{FFFFFF} \textbf{Promedio}}} \\
\cellcolor[HTML]{C0C0C0}{\color[HTML]{333333} a280.tsp}                            & 280                                                                                 & 3394                                                                             & 3329                                                                              & 3361.5                                                                               & 3396                                                                             & 3335                                                                              & 3365.5                                                                               & 3418                                                                             & 3318                                                                              & 3368                                                                                 \\
\cellcolor[HTML]{C0C0C0}{\color[HTML]{333333} brd14051.tsp}                        & 14051                                                                               & 620546                                                                           & 618878                                                                            & 619712                                                                               & 620743                                                                           & 619060                                                                            & 619901.5                                                                             & 622688                                                                           & 621273                                                                            & 621980.5                                                                             \\
\cellcolor[HTML]{C0C0C0}{\color[HTML]{333333} ch150.tsp}                           & 150                                                                                 & 8535                                                                             & 8272                                                                              & 8403.5                                                                               & 8540                                                                             & 8315                                                                              & 8427.5                                                                               & 8427                                                                             & 8231                                                                              & 8329                                                                                 \\
\cellcolor[HTML]{C0C0C0}{\color[HTML]{333333} d1655.tsp}                           & 1655                                                                                & 82881                                                                            & 82098                                                                             & 82489.5                                                                              & 82839                                                                            & 82014                                                                             & 82426.5                                                                              & 82447                                                                            & 81996                                                                             & 82221.5                                                                              \\
\cellcolor[HTML]{C0C0C0}{\color[HTML]{333333} d493.tsp}                            & 493                                                                                 & 45012                                                                            & 44124                                                                             & 44568                                                                                & 45275                                                                            & 44054                                                                             & 44664.5                                                                              & 44918                                                                            & 43373                                                                             & 44145.5                                                                              \\
\cellcolor[HTML]{C0C0C0}{\color[HTML]{333333} eil101.tsp}                          & 101                                                                                 & 822                                                                              & 788                                                                               & 805                                                                                  & 825                                                                              & 788                                                                               & 806.5                                                                                & 802                                                                              & 781                                                                               & 791.5                                                                                \\
\cellcolor[HTML]{C0C0C0}{\color[HTML]{333333} fl417.tsp}                           & 417                                                                                 & 17070                                                                            & 16672                                                                             & 16871                                                                                & 17028                                                                            & 16610                                                                             & 16819                                                                                & 16709                                                                            & 16450                                                                             & 16579.5                                                                              \\
\cellcolor[HTML]{C0C0C0}{\color[HTML]{333333} lin318.tsp}                          & 318                                                                                 & 55085                                                                            & 53752                                                                             & 54418.5                                                                              & 55073                                                                            & 53773                                                                             & 54423                                                                                & 54344                                                                            & 53479                                                                             & 53911.5                                                                              \\
\cellcolor[HTML]{C0C0C0}{\color[HTML]{333333} p654.tsp}                            & 654                                                                                 & 46767                                                                            & 45753                                                                             & 46260                                                                                & 46541                                                                            & 45588                                                                             & 46064.5                                                                              & 46292                                                                            & 45645                                                                             & 45968.5                                                                              \\
\cellcolor[HTML]{C0C0C0}{\color[HTML]{333333} pcb3038.tsp}                         & 3038                                                                                & 177468                                                                           & 176488                                                                            & 176978                                                                               & 177403                                                                           & 176282                                                                            & 176842.5                                                                             & 177957                                                                           & 176391                                                                            & 177174                                                                               \\
\cellcolor[HTML]{C0C0C0}{\color[HTML]{333333} rd400.tsp}                           & 400                                                                                 & 18922                                                                            & 18562                                                                             & 18742                                                                                & 18919                                                                            & 18559                                                                             & 18739                                                                                & 18790                                                                            & 18510                                                                             & 18650                                                                                \\
\cellcolor[HTML]{C0C0C0}{\color[HTML]{333333} rl5934.tsp}                          & 5934                                                                                & 777479                                                                           & 772530                                                                            & 775004.5                                                                             & 778205                                                                           & 772083                                                                            & 775144                                                                               & 779039                                                                           & 776338                                                                            & 777688.5                                                                             \\
\cellcolor[HTML]{C0C0C0}{\color[HTML]{333333} u159.tsp}                            & 159                                                                                 & 55996                                                                            & 55996                                                                             & 55996                                                                                & 56003                                                                            & 56003                                                                             & 56003                                                                                & 55174                                                                            & 55174                                                                             & 55174                                                                                \\
\cellcolor[HTML]{C0C0C0}{\color[HTML]{333333} u724.tsp}                            & 724                                                                                 & 56479                                                                            & 55514                                                                             & 55996.5                                                                              & 56725                                                                            & 55474                                                                             & 56099.5                                                                              & 55283                                                                            & 54685                                                                             & 54984                                                                                \\
\cellcolor[HTML]{C0C0C0}{\color[HTML]{333333} vm1084.tsp}                          & 1084                                                                                & 338557                                                                           & 330999                                                                            & 334778                                                                               & 339023                                                                           & 331589                                                                            & 335306                                                                               & 333749                                                                           & 329200                                                                            & 331474.5                                                                            
\end{tabular}
}
}
\label{table:ExperimentosConCuadrantes}
\end{table} 

\subsubsection{Experimentos sin aplicar el método de cuadrantes}
En la tabla \ref{table:ExperimentosSinCuadrantes} se usó las mismas columnas, esta vez muestra los datos de los experimentos sin haber aplicado el método de cuadrantes antes, como se puede notar las cantidades presentadas son mucho más altas que las anteriores. También de las 100 corridas que se realizó sobre cada problema se tomó la mejor solución, la peor y se hizo un promedio de todos los resultados.


% Please add the following required packages to your document preamble:
% \usepackage[table,xcdraw]{xcolor}
% If you use beamer only pass "xcolor=table" option, i.e. \documentclass[xcolor=table]{beamer}
\begin{table}[hbtp]
\centering
\caption{Experimentos sin aplicar método de cuadrantes.}
\resizebox{1\textwidth}{!}{
\rotatebox{0}{
\begin{tabular}{lrrrrrrrrrr}
\multicolumn{2}{c}{}                                                                                                                                                     & \multicolumn{9}{c}{\cellcolor[HTML]{343434}{\color[HTML]{FFFFFF} \textbf{Experimentos sin aplicar método de cuadrantes}}}                                                                                                                                                                                                                                                                                                                                                                                                                                                                                                                                                                                                                                                                                              \\
\multicolumn{2}{l}{}                                                                                                                                                     & \multicolumn{3}{c}{\cellcolor[HTML]{656565}{\color[HTML]{FFFFFF} \textbf{\begin{tabular}[c]{@{}c@{}}Recocido Simulado\end{tabular}}}}                                                                                                                   & \multicolumn{3}{c}{\cellcolor[HTML]{656565}{\color[HTML]{FFFFFF} \textbf{Greedy}}}                                                                                                                                                                          & \multicolumn{3}{c}{\cellcolor[HTML]{656565}{\color[HTML]{FFFFFF} \textbf{Genético}}}                                                                                                                                                                        \\
\rowcolor[HTML]{9B9B9B} 
\multicolumn{1}{c}{\cellcolor[HTML]{9B9B9B}{\color[HTML]{FFFFFF} \textbf{Nombre}}} & \multicolumn{1}{c}{\cellcolor[HTML]{9B9B9B}{\color[HTML]{FFFFFF} \textbf{Ciudades}}} & \multicolumn{1}{c}{\cellcolor[HTML]{9B9B9B}{\color[HTML]{FFFFFF} \textbf{Peor}}} & \multicolumn{1}{c}{\cellcolor[HTML]{9B9B9B}{\color[HTML]{FFFFFF} \textbf{Mejor}}} & \multicolumn{1}{c}{\cellcolor[HTML]{9B9B9B}{\color[HTML]{FFFFFF} \textbf{Promedio}}} & \multicolumn{1}{c}{\cellcolor[HTML]{9B9B9B}{\color[HTML]{FFFFFF} \textbf{Peor}}} & \multicolumn{1}{c}{\cellcolor[HTML]{9B9B9B}{\color[HTML]{FFFFFF} \textbf{Mejor}}} & \multicolumn{1}{c}{\cellcolor[HTML]{9B9B9B}{\color[HTML]{FFFFFF} \textbf{Promedio}}} & \multicolumn{1}{c}{\cellcolor[HTML]{9B9B9B}{\color[HTML]{FFFFFF} \textbf{Peor}}} & \multicolumn{1}{c}{\cellcolor[HTML]{9B9B9B}{\color[HTML]{FFFFFF} \textbf{Mejor}}} & \multicolumn{1}{c}{\cellcolor[HTML]{9B9B9B}{\color[HTML]{FFFFFF} \textbf{Promedio}}} \\
\cellcolor[HTML]{C0C0C0}{\color[HTML]{333333} a280.tsp}                            & 280                                                                                 & 5345                                                                             & 4711                                                                              & 5028                                                                                 & 5410                                                                             & 4804                                                                              & 5107                                                                                 & 5066                                                                             & 4747                                                                              & 4907                                                                                 \\
\cellcolor[HTML]{C0C0C0}{\color[HTML]{333333} brd14051.tsp}                        & 14051                                                                               & 1.35E+07                                                                         & 1.26E+07                                                                          & 1.30E+07                                                                             & 1.34E+07                                                                         & 1.28E+07                                                                          & 1.31E+07                                                                             & 2.35E+07                                                                         & 2.34E+07                                                                          & 2.35E+07                                                                             \\
\cellcolor[HTML]{C0C0C0}{\color[HTML]{333333} ch150.tsp}                           & 150                                                                                 & 28674                                                                            & 23162                                                                             & 25918                                                                                & 29187                                                                            & 22417                                                                             & 25802                                                                                & 37844                                                                            & 32091                                                                             & 34967.5                                                                              \\
\cellcolor[HTML]{C0C0C0}{\color[HTML]{333333} d1655.tsp}                           & 1655                                                                                & 243750                                                                           & 222473                                                                            & 233111.5                                                                             & 246172                                                                           & 226740                                                                            & 236456                                                                               & 266415                                                                           & 262348                                                                            & 264381.5                                                                             \\
\cellcolor[HTML]{C0C0C0}{\color[HTML]{333333} d493.tsp}                            & 493                                                                                 & 102751                                                                           & 91468                                                                             & 97109.5                                                                              & 100955                                                                           & 89956                                                                             & 95455.5                                                                              & 124989                                                                           & 115991                                                                            & 120490                                                                               \\
\cellcolor[HTML]{C0C0C0}{\color[HTML]{333333} eil101.tsp}                          & 101                                                                                 & 1917                                                                             & 1585                                                                              & 1751                                                                                 & 1855                                                                             & 1630                                                                              & 1742.5                                                                               & 1740                                                                             & 1602                                                                              & 1671                                                                                 \\
\cellcolor[HTML]{C0C0C0}{\color[HTML]{333333} fl417.tsp}                           & 417                                                                                 & 50870                                                                            & 46858                                                                             & 48864                                                                                & 50570                                                                            & 46460                                                                             & 48515                                                                                & 53761                                                                            & 50896                                                                             & 52328.5                                                                              \\
\cellcolor[HTML]{C0C0C0}{\color[HTML]{333333} lin318.tsp}                          & 318                                                                                 & 118920                                                                           & 102042                                                                            & 110481                                                                               & 117441                                                                           & 104454                                                                            & 110947.5                                                                             & 125074                                                                           & 114461                                                                            & 119767.5                                                                             \\
\cellcolor[HTML]{C0C0C0}{\color[HTML]{333333} p654.tsp}                            & 654                                                                                 & 127226                                                                           & 118683                                                                            & 122954.5                                                                             & 126211                                                                           & 118676                                                                            & 122443.5                                                                             & 135144                                                                           & 131454                                                                            & 133299                                                                               \\
\cellcolor[HTML]{C0C0C0}{\color[HTML]{333333} pcb3038.tsp}                         & 3038                                                                                & 396772                                                                           & 384530                                                                            & 390651                                                                               & 398492                                                                           & 382955                                                                            & 390723.5                                                                             & 427722                                                                           & 422981                                                                            & 425351.5                                                                             \\
\cellcolor[HTML]{C0C0C0}{\color[HTML]{333333} rd400.tsp}                           & 400                                                                                 & 118554                                                                           & 101860                                                                            & 110207                                                                               & 117363                                                                           & 102698                                                                            & 110030.5                                                                             & 181970                                                                           & 174218                                                                            & 178094                                                                               \\
\cellcolor[HTML]{C0C0C0}{\color[HTML]{333333} rl5934.tsp}                          & 5934                                                                                & 9492444                                                                          & 9184381                                                                           & 9338412.5                                                                            & 9497166                                                                          & 9240374                                                                           & 9368770                                                                              & 1.03E+07                                                                         & 1.02E+07                                                                          & 1.02E+07                                                                             \\
\cellcolor[HTML]{C0C0C0}{\color[HTML]{333333} u159.tsp}                            & 159                                                                                 & 72895                                                                            & 72895                                                                             & 72895                                                                                & 73170                                                                            & 73170                                                                             & 73170                                                                                & 64946                                                                            & 64946                                                                             & 64946                                                                                \\
\cellcolor[HTML]{C0C0C0}{\color[HTML]{333333} u724.tsp}                            & 724                                                                                 & 149630                                                                           & 134129                                                                            & 141879.5                                                                             & 149355                                                                           & 132686                                                                            & 141020.5                                                                             & 171946                                                                           & 164770                                                                            & 168358                                                                               \\
\cellcolor[HTML]{C0C0C0}{\color[HTML]{333333} vm1084.tsp}                          & 1084                                                                                & 3380496                                                                          & 3064309                                                                           & 3222402.5                                                                            & 3363287                                                                          & 3063644                                                                           & 3213465.5                                                                            & 4850260                                                                          & 4671155                                                                           & 4760707.5                                                                           
\end{tabular}
}
}
\label{table:ExperimentosSinCuadrantes}
\end{table} 

\subsubsection{Análisis global de los mejores resultados}
Por último en la tabla \ref{table:ExperimentosComparativos} se muestra la metaheurística del mejor resultado de las 3 (con cuadrantes y sin cuadrante) por cada problema expuesto, además de compararse con los mejores resultados (benchmark) obtenidos. Esta tabla está formada por 4 grupos que se explicará de izquierda a derecha:

\begin{itemize}
    \item \textbf{Con cuadrantes: }Aquí se muestra por cada problema de TSP, el mejor resultado obtenido de entre las 3 metaheurísticas empleadas en la tabla \ref{table:ExperimentosConCuadrantes}. Está dividido en las siguientes columnas:
    \begin{itemize}
        \item \textbf{Metaheurística: }El nombre de la metaheurística que produjo el mejor resultado.
        \item \textbf{Resultado: }La puntuación obtenida a través de dicha metaheurística.
    \end{itemize}
    \item \textbf{Con cuadrantes: }Aquí se muestra por cada problema de TSP, el mejor resultado obtenido de entre las 3 metaheurísticas empleadas en la tabla \ref{table:ExperimentosSinCuadrantes}. Está dividido en las siguientes columnas.
    \begin{itemize}
        \item \textbf{Metaheurística: }El nombre de la metaheurística que trajo el mejor resultado.
        \item \textbf{Resultado: }La puntuación obtenida a través de dicha metaheurística.
    \end{itemize}
    \item \textbf{Otros Datos: } Aquí se muestran información adicional:
    \begin{itemize}
        \item \textbf{Solo M.C: }Muestra el resultado que se obtuvo solo con el método de cuadrantes.
        \item \textbf{Benchmark: }Muestra el resultado de los benchmarks.
    \end{itemize}
    \item \textbf{Comparaciones con los benchmarks: } Por último aquí se hace una comparación del resultado de los benchmark que se puede ver en la columna de Otros datos, con los resultados de las columnas anteriores. El valor de la columna es la diferencia de ambos valores mostrados en forma de porcentaje.
    \begin{itemize}
        \item \textbf{Con cuadrantes: }Muestra la diferencia con el mejor resultado aplicando el método de cuadrantes.
        \item \textbf{Sin cuadrantes: }Muestra la diferencia con el mejor resultado sin aplicar el método de cuadrantes.
        \item \textbf{Solo M.C: }Muestra la diferencia con el resultado que se obtuvo solo con el método de cuadrantes.
    \end{itemize}
\end{itemize}
    
% Please add the following required packages to your document preamble:
% \usepackage[table,xcdraw]{xcolor}
% If you use beamer only pass "xcolor=table" option, i.e. \documentclass[xcolor=table]{beamer}
\begin{table}[hbtp]
\centering
\caption{Análisis global de los mejores resultados.}
\resizebox{1\textwidth}{!}{
\rotatebox{0}{
\begin{tabular}{lrlrlrrrrrr}
                                     &                                & \multicolumn{9}{c}{\cellcolor[HTML]{343434}{\color[HTML]{FFFFFF} Análisis global de los mejores resultados}}                                                                                                                                                                                                                                                                                                                                                                                                                                 \\
                                     &                                & \multicolumn{2}{c}{\cellcolor[HTML]{656565}{\color[HTML]{FFFFFF} \begin{tabular}[c]{@{}l@{}}Con Cuadrantes\end{tabular}}} & \multicolumn{2}{c}{\cellcolor[HTML]{656565}{\color[HTML]{FFFFFF}Sin Cuadrantes}} &\multicolumn{2}{c}{\cellcolor[HTML]{656565}{\color[HTML]{FFFFFF}Otros datos}} & \multicolumn{3}{c}{\cellcolor[HTML]{656565}{\color[HTML]{FFFFFF} Comparaciones entre el Benchmark (\%)}}                                                                                                                \\
\rowcolor[HTML]{9B9B9B} 
{\color[HTML]{FFFFFF} Nombre}        & {\color[HTML]{FFFFFF} Ciudades} & {\color[HTML]{FFFFFF} Metaheurística}                                                                                       & {\color[HTML]{FFFFFF} Resultado}                                                & {\color[HTML]{FFFFFF} Metaheurística}                           & {\color[HTML]{FFFFFF} Resultado} & {\color[HTML]{FFFFFF} Solo M.C.} & {\color[HTML]{FFFFFF} Benchmark} & {\color[HTML]{FFFFFF} Con Cuadrantes} & {\color[HTML]{FFFFFF} Sin Cuadrantes} & {\color[HTML]{FFFFFF} Solo M.C.} \\
\cellcolor[HTML]{C0C0C0}a280.tsp     & 280                            & Genético                                                                                                                    & 3318                                                                            & Recocido Simulado                                               & 4711                             & 3418                            & 2579                             & 28.65                               	  & 82.66                                 & 32.53                         \\
\cellcolor[HTML]{C0C0C0}brd14051.tsp & 14051                          & Recocido Simulado                                                                                                           & 618878                                                                          & Recocido Simulado                                               & 12612710                         & 623324                          & 469385                           & 31.84                                   & 2587.07                               & 32.79                         \\
\cellcolor[HTML]{C0C0C0}ch150.tsp    & 150                            & Genético                                                                                                                    & 8231                                                                            & Greedy                                                          & 22417                            & 8579                            & 6528                             & 26.08                                   & 243.39                                & 31.41                         \\
\cellcolor[HTML]{C0C0C0}d1655.tsp    & 1655                           & Genético                                                                                                                    & 81996                                                                           & Recocido Simulado                                               & 222473                           & 83605                           & 62128                            & 31.97                                   & 258.08                                & 34.56                         \\
\cellcolor[HTML]{C0C0C0}d493.tsp     & 493                            & Genético                                                                                                                    & 43373                                                                           & Greedy                                                          & 89956                            & 45731                           & 35002                            & 23.91                                   & 157.00                                & 30.65                         \\
\cellcolor[HTML]{C0C0C0}eil101.tsp   & 101                            & Genético                                                                                                                    & 781                                                                             & Recocido Simulado                                               & 1585                             & 828                             & 629                              & 24.16                                   & 151.98                                & 31.63                         \\
\cellcolor[HTML]{C0C0C0}fl417.tsp    & 417                            & Genético                                                                                                                    & 16450                                                                           & Greedy                                                          & 46460                            & 17419                           & 11861                            & 38.68                                   & 291.70                                & 46.85                         \\
\cellcolor[HTML]{C0C0C0}lin318.tsp   & 318                            & Genético                                                                                                                    & 53479                                                                           & Recocido Simulado                                               & 102042                           & 55340                           & 42029                            & 27.24                                   & 142.78                                & 31.67                         \\
\cellcolor[HTML]{C0C0C0}p654.tsp     & 654                            & Greedy                                                                                                                      & 45588                                                                           & Greedy                                                          & 118676                           & 47464                           & 34643                            & 31.59                                   & 242.56                                & 37.00                         \\
\cellcolor[HTML]{C0C0C0}pcb3038.tsp  & 3038                           & Greedy                                                                                                                      & 176282                                                                          & Greedy                                                          & 382955                           & 179474                          & 137694                           & 28.02                                   & 178.12                                & 30.34                         \\
\cellcolor[HTML]{C0C0C0}rd400.tsp    & 400                            & Genético                                                                                                                    & 18510                                                                           & Recocido Simulado                                               & 101860                           & 19094                           & 15281                            & 21.13                                   & 566.57                                & 24.95                         \\
\cellcolor[HTML]{C0C0C0}rl5934.tsp   & 5934                           & Greedy                                                                                                                      & 772083                                                                          & Recocido Simulado                                               & 9184381                          & 785540                          & 556045                           & 38.85                                   & 1551.73                               & 41.27                         \\
\cellcolor[HTML]{C0C0C0}u159.tsp     & 159                            & Genético                                                                                                                    & 55174                                                                           & Genético                                                        & 64946                            & 56495                           & 42080                            & 31.11                                   & 54.33                                 & 34.25                         \\
\cellcolor[HTML]{C0C0C0}u724.tsp     & 724                            & Genético                                                                                                                    & 54685                                                                           & Greedy                                                          & 132686                           & 57641                           & 41910                            & 30.48                                   & 216.59                                & 37.53                         \\
\cellcolor[HTML]{C0C0C0}vm1084.tsp   & 1084                           & Genético                                                                                                                    & 329200                                                                          & Greedy                                                          & 3063644                          & 344972                          & 239297                           & 37.56                                   & 1180.26                               & 44.16                        
\end{tabular}
}
}
\label{table:ExperimentosComparativos}
\end{table} 
        
\clearpage \newpage