\chapter{}
    \vspace*{10cm}
    \begin{center}
        %\thispagestyle{empty} 
        \section*{Conclusión}
        \addcontentsline{toc}{section}{CONCLUSIÓN}
        \addtocounter{section}{1}
        \setcounter{subsection}{0}
        \setcounter{figure}{0}
        \setcounter{lstlisting}{0}
        \setcounter{table}{0} 
    \end{center}
    
    \clearpage \newpage
    \vspace*{8cm}
    
La aplicación de un algoritmo que divide un problema de TSP en cuadrantes logró obtener soluciones lo suficientemente eficientes para el tiempo de ejecución, además de que el uso de metaheurísticas para modificar el resultado anterior permitió mejorar la solución inicial.\\
\hspace*{1cm}Después de realizar los experimentos quedo demostrado que el uso de algoritmos determínistas en conjunto con alguna metaheurística permite obtener resultados cercanos al óptimo, aunque no superan los resultados obtenidos en los benchmark, si son lo suficientemente eficientes para presentar un gran avance.\\
\hspace*{1cm}El conjunto de problemas usados en la prueba (TSPLIB y arte) y los resultados obtenidos demuestran que es factible aplicar esta técnica para resolver el problema del agente viajero, con lo cual la hipótesis queda demostrada.\\
\hspace*{1cm}Otro objetivo que se esperaba de este trabajo es que pueda ser utilizado a futuro en otros trabajos similares de optimización combinatoria como el problema de ruteo de vehículos y el problema de asignación de trabajos (scheduling).\\
\hspace*{1cm}Como ejemplo está el tema \ref{subsection:ARTTSP} (Arte de TSP) que muestra el uso los algoritmos del agente viajero para crear imágenes, incluyendo usar dicha metodología para otras funciones como el renderizado de imágenes al redistribuir los puntos de un determinado cuadrante, aunque no se abarco este tema a profundidad es posible usar el método de aplicación de cuadrantes para poder realizar dicha función.\\
%\hspace*{1cm}Como últimos comentarios acerca de este trabajo, tanto el trabajo escrito como el programa usado como herramienta para realizar los experimentos comenzó desde inicios del 2013 y tomó más de 4 años en desarrollarse.
\newpage
