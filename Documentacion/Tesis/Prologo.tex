\section*{Resumen}
%El tema de este trabajo es acerca de la aplicación de un algoritmo que sigue un mismo camino, seguido de otro que se dedica a la selección de valores aleatorios. La combinación de ambas tecnicas permite crear mejores resultados que las de sus predecesores, donde cada nuevo "espécimen" sea probado, replicado o desechado según sea el caso.\\
%\hspace*{1cm}El objetivo de un algoritmo de inteligencia artificial (IA) es obtener un resultado a través de una serie de tareas y patrones establecidos para reaccionar ante cualquier situación. Sin embargo, también es necesario que exista ese factor de 'suerte', donde la IA pueda ir más allá de los esperado tomando caminos al azar, pero sin ignorar su concepto de sentido común, tal como hacen los seres humanos al tomar decisiones.\\
\hspace*{1cm}Este trabajo de tesis describe el desarrollo de una técnica para encontrar una solución al problema del agente viajero, que es una referencia clásica en el área de la optimización combinatoria y se utiliza como un caso de prueba estándar para evaluar la efectividad de diversos métodos de optimización.\\
\hspace*{1cm}La técnica propuesta en este documento utiliza una agrupación de soluciones parciales de un problema en cuadrantes del espacio euclidiano con el fin de obtener una solución cercana al óptimo a través de una regla determinística.\\
\hspace*{1cm}La solución inicial encontrada a través de este método es refinada en una etapa subsecuente utilizando un conjunto de metaheurísticas. Los resultados obtenidos indican que el uso de esta estrategia permite encontrar soluciones cercanas al óptimo para el problema del agente viajero.\\
\hspace*{1cm}En \cite{[REF1]} se indica que el problema del agente viajero es posiblemente el más importante de los problemas de optimización combinatoria ya que fue uno de los primeros en ser clasificados como “duros” técnicamente hablando.\\
\hspace*{1cm} El texto está dividido en los siguientes capítulos:
\begin{itemize}
\item \textbf{Marco metodológico: }El marco metodológico es el inicio de cada investigación, es una breve introducción que plantea el documento: la descripción del problema, la hipótesis propuesta, los objetivos que pretende alcanzar, las justificaciones de realizarlo, así como los alcances y las limitaciones que tiene del trabajo. En este caso se explicará la problemática del TSP y que tipo de solución se usará para resolver dicho problema.
\item \textbf{Marco teórico: }Para una mejor comprensión del tema, en este capítulo se describen todos los temas que fueron usados para el desarrollo de esta tesis, desde la definición del problema del agente viajero, así como de conceptos como metaheurística, la teoría de autómatas, etc. La intención de este capítulo es informar al lector y que pueda comprender la solución implementada en los siguientes capítulos. La información obtenida para el desarrollo de este capítulo es respaldada por fuentes de diversas instituciones.
\item \textbf{Método de agrupamiento basado en cuadrantes: }Éste es el nombre que se le dio al algoritmo para resolver las instancias del problema del agente viajero y en este capítulo se explica la forma en que trabaja. Básicamente, este algoritmo usa técnicas de agrupamiento para dividir el problema en otros más pequeños con la intención de reducir la explosión combinatoria del problema.
\item \textbf{Documentación del software usado en las pruebas: }En este capítulo se presenta una descripción del programa que se implementó en esta tesis. Se explica de que elementos se compone la interfaz, como funciona y el papel que desempeñan las metaheurísticas. Al final de este capítulo se hace una demostración de cómo se utilizó este programa para el desarrollo de los experimentos descritos en el siguiente capítulo.
\item \textbf{Pruebas y análisis de resultados: }Por último se usará el programa descrito en el capítulo anterior para realizar experimentos sobre varias instancias del problema del agente viajero conocidos dentro del medio. Estos problemas fueron obtenidos en la página de la Universidad de Heidelberg (\cite{[TSPLIB]}). Una vez terminado los experimentos se construyeron gráficas y se compararon los resultados obtenidos con los reportados en la literatura. Adicionalmente se aplica el método propuesto en este trabajo para resolver un grupo de problemas conocidos como "Arte con TSP". Una vez finalizado los capítulos, se presentan las conclusiones de este trabajo, así como las referencias bibliográficas. 
\end{itemize}
\newpage