\section*{Conclusión}

El haber desarrollado este trabajo, tanto la obra escrita como el programa realizado para su desarrollo fue sin duda una de las experiencias mas satisfactorias de mi vida profesional donde he puesto a prueba todo lo que he aprendido, dentro y fuera de la universidad.\\
\hspace*{1cm}En este trabajo se desarrollo una técnica basada en cuadrantes con el fin de crear un método fácil de replicar usando el sentido común, el uso de metaheuristicas permite obtener una variedad diferente de soluciones derivadas del método anterior mas eficientes y capaces de mejorarse así mismos.\\
\hspace*{1cm}El conjunto de problemas usados en la prueba (TSPLIB y arte) y los resultados obtenidos demuestran que es factible aplicar esta técnica para resolver el problema del agente viajero, con lo cual la hipótesis queda demostrada.\\
\hspace*{1cm}Otro objetivo que se esperaba de este trabajo es que pueda ser utilizado a futuro en otros trabajos similares de optimización combinatoria como el problema de ruteo de vehículos y el problema de asignación de trabajos (scheduling).\\
\hspace*{1cm}La redacción de este escrito tomo mas de 3 años de redacción, desarrollo y prueba para otorgar un trabajo lo suficientemente convincente de poder aportar mi pequeño grano de arena al conocimiento humano, esperando que pueda ser usado en futuras generaciones e ir mas allá de lo que yo logre aquí. Esperando que esta tesis sea la llave para una infinidad de posibilidades tanto para mí como para aquel involucrado.\\\\
\hspace*{1cm}Con todo lo anterior me despido.\\\\\\
Atentamente:\\
Guillermo Sebastián Medina Palacios


\newpage
