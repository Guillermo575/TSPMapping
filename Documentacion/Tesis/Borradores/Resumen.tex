\section*{Resumen}
La tesis trata sobre el problema del agente viajero (también conocido como TSP) que es un problema computacional en el que trata de recorrer una cantidad específica de puntos en la menor distancia posible, debido a la cantidad exponencial de combinaciones posibles es considerado como un problema NP-Completo, eso quiere decir que no puede resolverse por completo usando algoritmos convencionales.\\
\hspace*{1cm} Aunque es posible obtener resultados óptimos mediante el uso de metaheuristicas que son fórmulas matemáticas que involucran números aleatorios para su funcionamiento, la cantidad de recursos necesarios para realizar dichas operaciones hace que lleve mucho tiempo poder obtener resultados, difícil que pueda obtenerse mediante una computadora ordinaria.\\
\hspace*{1cm} El objetivo de la tesis es usar un algoritmo de agrupamiento (clustering) que nos permita obtener un resultado coherente, rápido de ejecutar y sin usar medidas aleatorias para llegar a dicha respuesta, así ahorrándonos tiempo y recursos.\\
\hspace*{1cm} A pesar de lo anterior, también se planea usar las antes mencionadas metaheuristicas para hacer pequeñas modificaciones en los resultados obtenidos anteriormente, aunque presenten diferencias leves esto le dará más flexibilidad al proyecto porque se puede obtener diferentes respuestas sin salir drásticamente de la solución inicial.\\
\hspace*{1cm}El uso de metaheuristicas ayuda a demostrar que la inteligencia artificial se basa en tomar decisiones inesperadas sin salir de su lógica programada, que humanamente podría llamarse sentido común.\\
\hspace*{1cm} Para poder realizar estas tareas se desarrolló un programa creado por su servidor, escrito en JAVA que al leer unos archivos .tsp permitirá resolver el problema y mostrar de manera gráfica los resultados, los archivos .tsp son listas de puntos con un formato ya predeterminado, estas fueron hechas por otras universidades e instituciones que también trabajan en el tema del TSP como la universidad de Heidelberg o la universidad de Waterloo y dichos archivos se pueden descargar en sus respectivas páginas, también tienen registrados los mejores resultados obtenidos en cada problema.\\
\hspace*{1cm} Aunque no se pretende obtener el mejor resultado, se quiere aportar una nueva forma de resolver el problema de TSP y guardar tanto el resultado obtenido mediante un algoritmo como los otros que pueden generarse mediante el uso de metaheuristicas.\\
\hspace*{1cm} La tesis describe el desarrollo de una técnica para encontrar una solución del problema del agente viajero, que es una referencia clásica en el área de la optimización combinatoria y se utiliza como un caso de prueba estándar para evaluar la efectividad de diversos métodos de optimización. En \cite{[REF1]} se indica que el problema del agente viajero es probablemente el más importante entre los problemas de optimización combinatoria ya que fue uno de los primeros problemas de optimización clasificados como “duros” en un sentido técnico. La técnica propuesta en este trabajo utiliza una agrupación de soluciones parciales de un problema en cuadrantes del espacio euclidiano con el ánimo de encontrar una solución cercana al óptimo a través de una regla determinística. La solución inicial encontrada a través de este agrupamiento por cuadrantes es refinada en una etapa subsecuente utilizando un conjunto de metaheurísticas. Los resultados obtenidos indican que el uso de esta estrategia permite encontrar soluciones cercanas al óptimo para el problema del agente viajero, el texto está dividido en los siguientes capítulos:
\begin{itemize}
	\item \textbf{Marco metodológico: } Breve introducción que plantea el problema, los objetivos, justificaciones y limitaciones.
	\item \textbf{Marco téorico: } Aquí contiene toda la información recolectada para una mejor comprensión de lo que abarca el problema.
	\item \textbf{Método de agrupamiento basado en cuadrantes: } Este es el nombre que se le dio al algoritmo basado en agrupamiento, aquí se explica la forma en que trabaja.
	\item \textbf{Documentación del software usado en las pruebas: } Es un manual del programa que se hizo para este problema, también explica la forma en la que se utilizó para el desarrollo del siguiente capítulo.
	\item \textbf{Pruebas y análisis de resultados: } En este último capítulo se usó el software sobre varios problemas de TSP, se graficó y comparo el rendimiento con los mejores resultados obtenidos en la universidad de Heidelberg.
\end{itemize}
\newpage
