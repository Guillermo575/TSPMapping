https://www.ecured.cu/Aut%C3%B3mata_finito
https://ccc.inaoep.mx/~emorales/Cursos/Automatas/AutomatasFinitos.pdf
http://www.uhu.es/raul.jimenez/DIGITAL_II/dig2_ii.pdf
http://ocw.uc3m.es/ingenieria-informatica/teoria-de-automatas-y-lenguajes-formales/material-de-clase-1/tema-3-automatas-finitos
http://automatas-finitos.blogspot.mx/
http://www.ia.urjc.es/grupo/docencia/automatas_itis/apuntes/capitulo6.pdf
http://documents.tips/download/link/teoria-de-automatas-y-lenguajes-formales-pinocho-pdf
https://www.ecured.cu/Transformaci%C3%B3n_de_aut%C3%B3mata_finito_no_determinista_a_aut%C3%B3mata_finito_determinista


%                <Especies>
%                <Item>
%                    <genes>12</genes>
%                   <variantes>3</variantes>
%                   <br>
%                       <TipoRuta>EUC_2D</TipoRuta>
%                   </br>
%                  <removeUnable>true</removeUnable>
%                  <resultado>504.0</resultado>
%              </Item>
%              <Item>
%                  <genes>11</genes>
%                  <genes>12</genes>
%                  <variantes>4</variantes>
%                  <variantes>4</variantes>
%                  <br>
%                      <TipoRuta>EUC_2D</TipoRuta>
%                  </br>
%                 <removeUnable>true</removeUnable>
%                 <resultado>503.0</resultado>
%              </Item>
%              <Item>
%                  <genes>12</genes>
%                  <genes>14</genes>
%                 <genes>13</genes>
%                 <variantes>2</variantes>
%                 <variantes>2</variantes>
%                 <variantes>3</variantes>
%                <br>
%                     <TipoRuta>EUC_2D</TipoRuta>
%                 </br>
%                  <removeUnable>true</removeUnable>
%                  <resultado>502.0</resultado>
%              </Item>
%          </Especies>
    
    
%    ELEMENTOS:

%    -LOS GENES (ESPECIE)
% 	-MUTACION DE LA SOLUCION BASE
% 	-FORMA EN QUE SE APLICARA LAS METAHEURISTICAS PARA LA GENERACION DE MUTACIONES
	
% 	PROCESO:
	
% 	-tenemos la solución base
% 	-creamos una especie nueva compuesto por varios genes y valores base que actuaran como mutadores
% 	-primero se prueba si cada uno de los genes funciona tomando solo los que generan un resultado individual mejor
% 	-luego mutamos la solución base para ver los resultados
% 	-dependiendo del experimento se determina si es factible o no se usara

https://html5up.net/

https://mx1.gomsa.com/owa/auth/logon.aspx?replaceCurrent=1&url=https%3a%2f%2fmx1.gomsa.com%2fowa%2f
