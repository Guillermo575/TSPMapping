    \begin{lstlisting}[language=JAVA, caption=Pseudocodigo del Rectangulo aureo, label=lst:codigo10]

limite=5;     
clase punto{float x,y;} 

public Lista<punto> resolverProblema(Lista<punto> lstpuntos){
//El metodo principal, recibe una lista de puntos y el establece el origen y la prioridad
	Lista<punto> listanueva = resolverruta(lstpuntos,lstpuntos[0];new int[]{0,1,3,2});
	regresa listanueva;
}

public Lista<punto> resolverRuta(Lista<punto> lstpuntos,punto origen;int[] prioridad)){
	Lista<punto> listanueva;
/* En este metodo checa si el numero de puntos que recibe es menor del limite para resolverse, 
   de lo contrario se llama a otro metodo cuya funcion es fragmentar la lista */
	Si(lstpuntos.length<=limite){
		listanueva= fuerzaBruta(Lista<punto>);
	}else
		listanueva= resolverCuadrantes(lstpuntos,origen;prioridad);
	return listanueva;
}

public Lista<punto> resolverCuadrantes(Lista<punto> lstpuntos,punto origen;int[] prioridad)){
/* En este metodo se encargara de dividir la lista en 4 grupos de acuerdo a la relación con los ejes X,Y del punto origen, 
   una ves agrupados se llamara de nuevo al metodo de resolverRuta y a su ves llamara de nuevo este método en caso de necesitar 
   otra fragmentación una vez terminado se agruparan los 4 cuadrantes en uno solo y se iran integrando*/
	Lista<punto>[] cuadrantes= new Lista<punto>[4];
	Lista<punto> listanueva;
	/* Este ciclo se encarga de signar a que cuadrante pertenecera*/
	Ciclo(int l=0;l<lstpuntos.length;l++){
		Si(lstpuntos[l].x<=origen.x && lstpuntos[l].y<=origen.y)cuadrantes[0].agregar(lstpuntos[l]);
		Si(lstpuntos[l].x>origen.x && lstpuntos[l].y<=origen.y)cuadrantes[1].agregar(lstpuntos[l]);
		Si(lstpuntos[l].x<=origen.x && lstpuntos[l].y>origen.y)cuadrantes[2].agregar(lstpuntos[l]);;
		Si(lstpuntos[l].x>origen.x && lstpuntos[l].y>origen.y)cuadrantes[3].agregar(lstpuntos[l]);;
	}
	/*En este ciclo se llama de nuevo al metodo de resolverRuta para cada uno de los 4 grupos, dentro de ahi se creara un punto medio para el origen de 
	  cada cuadrante*/
	Ciclo(int l=0;l<prioridad.length;l++){
		if(cuadrantes[prioridad[l]].length>0){
			punto origencuadrante; //es el hipotetico punto por el cual van a partir los demas cuadrantes en caso de que necesite refragmentarse otra ves
			punto masX,menosX,masY,menosY;// para poder obtener dicho punto se buscara que numeros estan mas y menos cerca de los ejes X y Y y asi calcular las coordenadas
			Ciclo(int m=0;m<cuadrantes[prioridad[l]].length;m++){
			   //aqui se verifica si un punto un punto esta mas cerca o lejos del eje X o Y que el anterior, si el ciclo esta empezando se pondran por default
			   Si(cuadrantes[prioridad[l]][m].X <= menosX.X || m==0)menosX=cuadrantes[prioridad[l]][m];
			   Si(cuadrantes[prioridad[l]][m].X >= masX.X || m==0)masX=cuadrantes[prioridad[l]][m];
			   Si(cuadrantes[prioridad[l]][m].Y <= menosY.Y || m==0)menosY=cuadrantes[prioridad[l]][m];
			   Si(cuadrantes[prioridad[l]][m].Y >= masY.Y || m==0)masY=cuadrantes[prioridad[l]][m];
			   //aqui verifica si el punto principal del cuadrante esta mas cerca del origen, en caso que haya otro mas cerca se cambiaran de lugares
			   Si(formulaDistancia(origen,cuadrantes[prioridad[l]][m])<formulaDistancia(origen,cuadrantes[prioridad[l]][0])){
				punto temporal=cuadrantes[prioridad[l]][0];
				cuadrantes[prioridad[l]][0]=cuadrantes[prioridad[l]][m];
				cuadrantes[prioridad[l]][m]= temporal;
			   }
			}
			// ya obtenidos las variables masX,menosX,masY,menosY se establece las coordenadas del origen
			origencuadrante.x=(masX.x+menosX.x)/2;
			origencuadrante.y=(masY.y+menosY.y)/2;
			
			//ahora ya se llamara a la clase resolverRuta para obtener una lista
			Lista<punto> sumacuadrante =resolverRuta(cuadrantes[prioridad[l]],origencuadrante;prioridad));
			for(int n=0;n<sumacuadrante.length;n++){
				listanueva.agregar(sumacuadrante[0]);
			}
		}
	}
	retorno listanueva;
}

public Lista<punto> fuerzaBruta(Lista<punto> lstpuntos){
  Lista<punto> lstnueva;
  /*Aqui se crea combinaciones con todos los numeros sin repetir y usando todos con el fin de determinar en que orden es mas optimo, 
  al final devuelve una lista con los puntos ya ordenados*/
  retorna lstnueva;
}

public float formulaDistancia(punto inicio, punto fin){
  //usando la formula $v ((?x2?^2-?x1?^2 )+(?y2?^2-?y1?^2))$  para hallar la distancia.
}

/*NOTA : Para evitar complicar por el momento el codigo solo se esta utilizando un arreglo de prioridad, se tiene contemplando 
un metodo para averiguar que prioridad puede ser la mejor o bien, incluso usar un metodo de fuerza bruta solo que en ves de agrupar puntos serian listas*/ 

    \end{lstlisting}