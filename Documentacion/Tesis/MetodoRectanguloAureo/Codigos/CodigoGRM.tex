\begin{lstlisting}[language=C++, caption=Algoritmo Base del Método de Cuadrantes, label=lst:codigo11,escapechar=|]

public class GoldenRectangleMethod 
{  
	final int [][]izquierda = new int[][]
	{
		new int[]{0,1,2,3},
		new int[]{1,2,3,0}, 
		new int[]{2,1,0,3},
		new int[]{3,2,1,0},
	}; 
	final int [][]derecha = new int[][]
	{
		new int[]{0,3,2,1},
		new int[]{1,0,3,2}, 
		new int[]{2,3,0,1},
		new int[]{3,0,1,2},
	}; 
	final int [][]arriba=new int[][]
	{
		new int[]{0,3,2,1},
		new int[]{1,2,3,0}, 
		new int[]{2,3,0,1},
		new int[]{3,2,1,0},
	};    
	final int [][]abajo = new int[][]
	{
		new int[]{0,1,2,3},
		new int[]{1,0,3,2}, 
		new int[]{2,1,0,3},
		new int[]{3,0,1,2},
	};    
		
	public ArrayList<Coordenada> centros = new ArrayList<Coordenada>();
	public ArrayList<Coordenada> mins = new ArrayList<Coordenada>();     
	public ArrayList<Coordenada> maxs = new ArrayList<Coordenada>();
	public BruteForce br = new BruteForce();
	int limite=2; 

	public int[] getPriori(int cuadranteorigen,int origen, int destino)
	{
		int [] nuevo = new int[0];
		if((cuadranteorigen == 1 && destino==0) || (cuadranteorigen == 2 && destino == 3) || (destino < 0)||(destino > 3)) 
		{
			nuevo = izquierda[origen];		
		}
		if((cuadranteorigen == 0 && destino == 1)||(cuadranteorigen == 3 && destino == 2)) 
		{
			nuevo = derecha[origen];
		}
		if((cuadranteorigen == 2 && destino == 1)||(cuadranteorigen == 3 && destino == 0))
		{
			nuevo = arriba[origen];		
		}
		if((cuadranteorigen == 0 && destino == 3)||(cuadranteorigen == 1 && destino == 2))
		{
			nuevo = abajo[origen];		
		}
		return nuevo;
	}

	public ArrayList<Coordenada> start(MarkerDictionary md)
	{
		//Inicio 
		if(md.marcadores.size() > 0)
		{
			ArrayList<Coordenada> lstpuntos=md.marcadores;
			br=new BruteForce();
			br.TipoRuta = md.Type;
			ArrayList<Coordenada> resuelto = resolverRuta(lstpuntos,lstpuntos.get(0),-1,-1);
			return resuelto;
		}
		else 
		{
			return md.marcadores;
		}
	}

	public ArrayList<Coordenada> resolverRuta(ArrayList<Coordenada> lstpuntos,Coordenada destino,int cuadranteorigen, int cuadrantedestino)
	{
		Coordenada centro=new Coordenada(0, 0);
		double MinX = 0;
		double MaxX = 0;
		double MinY = 0;
		double MaxY = 0;     
		for(int l=0;l<lstpuntos.size();l++)
		{
            /* 
                Se buscan los extremos de cada cuadrante para establecer un punto centro, esto es en caso de que se necesite volver a fragmentar de nuevo el cuadrante 
            */
		   Coordenada punto=lstpuntos.get(l);
		   if(punto.x<MinX || l==0)	MinX = punto.x;
		   if(punto.x>MaxX || l==0)	MaxX = punto.x;
		   if(punto.y<MinY || l==0)	MinY = punto.y;
		   if(punto.y>MaxY || l==0)	MaxY = punto.y; 
		}
		centro.x = (MinX+MaxX)/2;
		centro.y = (MinY+MaxY)/2;
		centros.add(centro);
		mins.add(new Coordenada(MinX, MinY));
		maxs.add(new Coordenada(MaxX, MaxY));
		/* 
		    Si alcanza menos del limite se resuelve por el método de fuerza bruta, si no se fragmentara de manera infinita hasta satisfacer el limite 
		*/
		if(lstpuntos.size() <= limite)
		{
			return br.start(lstpuntos,destino);
		}
		else return resolverCuadrante(lstpuntos,cuadranteorigen,cuadrantedestino);
	}

	public ArrayList<Coordenada> resolverCuadrante(ArrayList<Coordenada> lstpuntos,int cuadranteorigen, int cuadrantedestino)
	{
		Coordenada centro = centros.get(centros.size()-1);
		Coordenada max = maxs.get(maxs.size()-1);
		Coordenada min = mins.get(mins.size()-1);
		ArrayList<Coordenada> total = new ArrayList<Coordenada>(); //Resultado final
		ArrayList<ArrayList<Coordenada>> cuadrantes = new ArrayList<ArrayList<Coordenada>>();
		ArrayList<Coordenada> destinos = new ArrayList<Coordenada>(4);
		int[]priori = new int[0];

		for(int l=0;l < 4;l++)
		{
            /* 
                Se crea los nuevos 4 cuadrantes junto con puntos intermedios para conectar cuadrante-cuadrante 
            */
			cuadrantes.add(new ArrayList<Coordenada>());
			destinos.add(new Coordenada(0,0));
		}

		for(int l=0;l<lstpuntos.size();l++)
		{
			/* Dependiendo de su posicion con respecto a los ejes del centro se iran a un determinado cuadrante */
			Coordenada punto=lstpuntos.get(l);
			if(punto.x <= centro.x && punto.y <= centro.y)
			{
				cuadrantes.get(0).add(punto);
				if(cuadranteorigen == -1)
				{
					cuadranteorigen = 0;
				}
			}
			else 
			if(punto.x > centro.x && punto.y <= centro.y)
			{
				cuadrantes.get(1).add(punto);
				if(cuadranteorigen==-1)
				{
					cuadranteorigen=1;
				}
			}
			else 
			if(punto.x <= centro.x && punto.y > centro.y)
			{
				cuadrantes.get(3).add(punto);
				if(cuadranteorigen==-1)
				{
					cuadranteorigen=3;
				}
			}
			else 
			if(punto.x > centro.x && punto.y > centro.y)
			{
				cuadrantes.get(2).add(punto);
				if(cuadranteorigen == -1)	
				{
					cuadranteorigen=2;
				}
			} 
			if(l == 0)
			{
				priori=getPriori(cuadranteorigen,cuadranteorigen,cuadrantedestino);
			}
		}             
		 
        /* 
            Debido a que se conectan al final los 4 cuadrantes sin relacionarse y el primer punto es el origen se crea unos puntos intermedios para verificar la cercanía que hay en el eje X o Y dependiendo de la cuadrante, el mas cercano se convierte en el primer punto 
        */           
		destinos.set(0,new Coordenada(centro.x,min.y));
		destinos.set(1,new Coordenada(max.x,centro.y));
		destinos.set(2,new Coordenada(centro.x,max.y));       
		destinos.set(3,new Coordenada(min.x,centro.y));    

        /* 
            Después de preparar dichos parámetros se llama otra ves al método resolverruta en cada cuadrante, el resultado se añadirá a la lista total, cabe decir que este cambio de métodos de resolverRuta y resolverCuadrante se pueden llegar a repetir hasta el infinito, hasta que satisfaga el limite 
        */             
		for(int l = 0; l < 4; l++)
		{
			int siguientedestino = l+1;
			if(siguientedestino>3)siguientedestino = 0;
			ArrayList<Coordenada> puntos = cuadrantes.get(priori[l]);
			if(puntos.size() > 0)
			{
				if(l==3)
				{
					switch(priori[3])
					{
						case 0: destinos.set(0,new Coordenada(min.x,min.y));
						case 1: destinos.set(1,new Coordenada(max.x,min.y));
						case 2: destinos.set(2,new Coordenada(max.x,max.y));                               
						case 3: destinos.set(3,new Coordenada(min.x,max.y));
						default : destinos.set(3,new Coordenada(centro.x,centro.y)); 
					}                    
				}
				int origencuadrante = priori[l];
				int siguiencuadrante = priori[siguientedestino];
				if(l == 3 && cuadrantedestino >= 0) 
				{
					origencuadrante = cuadranteorigen;
					siguiencuadrante = cuadrantedestino;
				}
				puntos=resolverRuta(puntos,destinos.get(priori[l]),origencuadrante,siguiencuadrante);
				for(int m=0;m<puntos.size();m++)
				{
					total.add(puntos.get(m));
				}   
				/* Busca el punto mas cercano del cuadrante */
				if(l<3)
				{
					Coordenada ultimo=puntos.get(puntos.size()-1);
					int cuadrantesiguiente=priori[l+1];
					if(cuadrantes.get(cuadrantesiguiente).isEmpty() &&l<2)
					{
						ultimo = new Coordenada(centro.x,centro.y);
						cuadrantesiguiente=priori[l+2];  
					}                     
					if(cuadrantes.get(cuadrantesiguiente).isEmpty() &&l<1)
					{
						ultimo = new Coordenada(centro.x,centro.y);
						cuadrantesiguiente=priori[l+3]; 
					}
					if(!cuadrantes.get(cuadrantesiguiente).isEmpty())
					{
						ArrayList<Coordenada> puntossiguientes=cuadrantes.get(cuadrantesiguiente); 
						double distanciamenor = br.getDistance(ultimo,puntossiguientes.get(0));
						int cursor = 0;                            
						Coordenada cambio = new Coordenada(puntossiguientes.get(cursor).x,puntossiguientes.get(cursor).y);  
						Coordenada cambio2 = new Coordenada(puntossiguientes.get(0).x, puntossiguientes.get(0).y);
						for(int m = 1; m < puntossiguientes.size(); m++)
						{
						  double distancianueva=br.getDistance(ultimo,puntossiguientes.get(m));
						  if(distancianueva < distanciamenor)
						  {                               
								distanciamenor = distancianueva;
								cursor = m;
								cambio = new Coordenada(puntossiguientes.get(0).x, puntossiguientes.get(0).y);
								cambio2 = new Coordenada(puntossiguientes.get(cursor).x,puntossiguientes.get(cursor).y);                       
						  }
						}
						puntossiguientes.get(0).setCoords(cambio2.x, cambio2.y);
						puntossiguientes.get(cursor).setCoords(cambio.x, cambio.y);                         
					}
				}   
			}
		}   
		return total;
	}	
}
\end{lstlisting}  