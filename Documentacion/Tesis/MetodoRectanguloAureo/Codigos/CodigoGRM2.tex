\begin{lstlisting}[language=C++, caption=Algoritmo Base del Método de Cuadrantes, label=lst:codigo11,escapechar=|]
final int [][]izquierda = new int[][]{new int[]{0,1,2,3}, new int[]{1,2,3,0}, new int[]{2,1,0,3}, new int[]{3,2,1,0},}; 
final int [][]derecha   = new int[][]{new int[]{0,3,2,1}, new int[]{1,0,3,2}, new int[]{2,3,0,1}, new int[]{3,0,1,2},}; 
final int [][]arriba    = new int[][]{new int[]{0,3,2,1}, new int[]{1,2,3,0}, new int[]{2,3,0,1}, new int[]{3,2,1,0},};    
final int [][]abajo     = new int[][]{new int[]{0,1,2,3}, new int[]{1,0,3,2}, new int[]{2,1,0,3}, new int[]{3,0,1,2},};
public ArrayList<Coordenada> centros=new ArrayList<Coordenada>();
public ArrayList<Coordenada> mins=new ArrayList<Coordenada>();     
public ArrayList<Coordenada> maxs=new ArrayList<Coordenada>();
public BruteForce br=new BruteForce();
int limite=2; //Si cada ruta a resolver excede este limite, se fragmentara por cuadrantes 
/*INICIO*/    
      public ArrayList<Coordenada> start(MarkerDictionary md){    return resolverRuta(md.marcadores,md.marcadores.get(0),-1,-1);    }
/*COMPROBACION*/ 
      public ArrayList<Coordenada> resolverRuta(ArrayList<Coordenada> lstpuntos,Coordenada destino,int cuadranteorigen, int cuadrantedestino){
               /*Se buscan los extremos de cada cuadrante para establecer un punto centro, esto es en caso de que se necesite volver a defragmentar de nuevo el cuadrante*/
                      double MinX=0;double MaxX=0;double MinY=0;double MaxY=0;               
                      for(int l=0;l<lstpuntos.size();l++){
                             Coordenada punto=lstpuntos.get(l);
                             if(punto.x<MinX || l==0)MinX=punto.x;
                             if(punto.x>MaxX || l==0)MaxX=punto.x;
                             if(punto.y<MinY || l==0)MinY=punto.y;
                             if(punto.y>MaxY || l==0)MaxY=punto.y; 
                      }
                      centros.add(new Coordenada((MinX+MaxX)/2, (MinY+MaxY)/2));
                      mins.add(new Coordenada(MinX, MinY));
                      maxs.add(new Coordenada(MaxX, MaxY));      
              /*Si se alcanza menos del limite se resuelve por el metodo de fuerza bruta, si no se fragmentara de manera infinita hasta satisfacer el limite*/
                    if(lstpuntos.size()<=limite) return br.start(lstpuntos,destino);
                    else return resolverCuadrante(lstpuntos,cuadranteorigen,cuadrantedestino);
      }
/*DIVISION*/     
      public ArrayList<Coordenada> resolverCuadrante(ArrayList<Coordenada> lstpuntos,int cuadranteorigen, int cuadrantedestino){
              Coordenada centro=centros.get(centros.size()-1);
              Coordenada max=maxs.get(maxs.size()-1);
              Coordenada min=mins.get(mins.size()-1);
              ArrayList<Coordenada> total=new ArrayList<Coordenada>(); //Resultado final
              ArrayList<ArrayList<Coordenada>> cuadrantes= new ArrayList<ArrayList<Coordenada>>();
              ArrayList<Coordenada> destinos=new ArrayList<Coordenada>(4);
              int[]priori=new int[0];
              /*Se crea los nuevos 4 cuadrantes junto con puntos intermedios para conectar cuadrante-cuadrante*/
                    for(int l=0;l<4;l++){
                          cuadrantes.add(new ArrayList<Coordenada>());
                          destinos.add(new Coordenada(0,0));
                    }
              /*DISTRIBUCION*//*Dependiendo de su posicion con respecto a los ejes del centro se iran a un determinado cuadrante*/
                    for(int l=0;l<lstpuntos.size();l++){
                          int nuevoCuadrante = 0;
                          Coordenada punto=lstpuntos.get(l);
                                if(punto.x<=centro.x && punto.y<=centro.y){nuevoCuadrante = 0;}
                           else if(punto.x>centro.x && punto.y<=centro.y){nuevoCuadrante = 1;}
                           else if(punto.x<=centro.x && punto.y>centro.y){nuevoCuadrante = 3;}
                           else if(punto.x>centro.x && punto.y>centro.y){nuevoCuadrante = 2;} 
                            cuadrantes.get(nuevoCuadrante).add(punto);
                            if(cuadranteorigen==-1)cuadranteorigen=nuevoCuadrante;
                            if(l==0)priori=getPriori(cuadranteorigen,nuevoCuadrante,cuadrantedestino);       
                    }   
               /*Debido a que se conectan al final los 4 cuadrantes sin relacionarse y el primer punto es el origen se crea unos puntos intermedios para verificar la cercania que hay en el eje X o Y dependiendo de la cuadrante*/           
                    destinos.set(0,new Coordenada(centro.x,min.y));
                    destinos.set(1,new Coordenada(max.x,centro.y));
                    destinos.set(2,new Coordenada(centro.x,max.y));       
                    destinos.set(3,new Coordenada(min.x,centro.y));   
                /*Aqui se elige cual va a ser el primer punto del siguiente cuadrante calculando la cercania que tiene con un punto elegido, de no tener puntos en las secciones vecinas se va al siguiente punto*/
                    for(int l=0;l<3;l++){
                        ArrayList<Coordenada> puntos=cuadrantes.get(priori[l]);
                          if(!puntos.isEmpty()){
                                  int cuadrantesiguiente = priori[l+1];
                                  Coordenada ultimo = puntos.get(puntos.size()-1);
                                  for(int m = l;m<2;m++){
                                        if(cuadrantes.get(cuadrantesiguiente).isEmpty() && l<2) {
                                                ultimo = new Coordenada(centro.x,centro.y);
                                                cuadrantesiguiente=priori[m+1]; 
                                        }
                                  }
                                  if(!cuadrantes.get(cuadrantesiguiente).isEmpty()){
                                          ArrayList<Coordenada> puntossiguientes=cuadrantes.get(cuadrantesiguiente); 
                                          double distanciamenor=br.getDistance(ultimo,puntossiguientes.get(0));
                                          Coordenada cambio= new Coordenada(puntossiguientes.get(0).x,puntossiguientes.get(0).y);  
                                          Coordenada cambio2= new Coordenada(puntossiguientes.get(0).x, puntossiguientes.get(0).y);                      
                                          for(int m=1;m<puntossiguientes.size();m++){
                                                  double distancianueva=br.getDistance(ultimo,puntossiguientes.get(m));
                                                  if(distancianueva<distanciamenor){                               
                                                        distanciamenor=distancianueva;
                                                        cambio=new Coordenada(puntossiguientes.get(0).x, puntossiguientes.get(0).y);
                                                        cambio2=new Coordenada(puntossiguientes.get(m).x,puntossiguientes.get(m).y);   
                                                        puntossiguientes.get(0).setCoords(cambio2.x, cambio2.y);
                                                        puntossiguientes.get(m).setCoords(cambio.x, cambio.y);                  
                                                  }
                                          }                       
                                  }
                         }
                    }
                /*Despues de preparar dichos parametros se llama otra ves al metodo resolverruta en cada cuadrante, el resultado se añadira a la lista total, se pueden llegar a repetir estas llamadas hasta que satisfaga el limite*/
                    for(int l=0;l<4;l++){
                              ArrayList<Coordenada> puntos=cuadrantes.get(priori[l]);
                              if(!puntos.isEmpty()){
                                      int siguientedestino=(siguientedestino>3)? 0 : (l+1);
                                      int origencuadrante=priori[l];
                                      int siguiencuadrante=priori[siguientedestino];
                                      if(l==3){
                                                switch(priori[3]){
                                                    case 0: destinos.set(0,new Coordenada(min.x,min.y));
                                                    case 1: destinos.set(1,new Coordenada(max.x,min.y));
                                                    case 2: destinos.set(2,new Coordenada(max.x,max.y));                               
                                                    case 3: destinos.set(3,new Coordenada(min.x,max.y));
                                                    default : destinos.set(3,new Coordenada(centro.x,centro.y)); 
                                                }      
                                                if(cuadrantedestino>=0) {
                                                        origencuadrante = cuadranteorigen;
                                                        siguiencuadrante = cuadrantedestino;
                                                }                                                    
                                      }    
                                      puntos = resolverRuta(puntos,destinos.get(priori[l]),origencuadrante,siguiencuadrante);                  
                                      for(int m=0;m<puntos.size();m++)  total.add(puntos.get(m));
                              }
                    }
              return total;
      }
/*OBTENER CUADRANTE ESPECIFICO*/    
      public int[] getPriori(int cuadranteorigen,int origen, int destino){
          if((cuadranteorigen==1 && destino==0)||(cuadranteorigen==2 && destino==3)|| (destino<0)||(destino>3))return izquierda[origen];
          if((cuadranteorigen==0 && destino==1)||(cuadranteorigen==3 && destino==2))return derecha[origen];
          if((cuadranteorigen==2 && destino==1)||(cuadranteorigen==3 && destino==0))return arriba[origen];
          if((cuadranteorigen==0 && destino==3)||(cuadranteorigen==1 && destino==2))return abajo[origen];
          return new int[0];
      }
\end{lstlisting} 