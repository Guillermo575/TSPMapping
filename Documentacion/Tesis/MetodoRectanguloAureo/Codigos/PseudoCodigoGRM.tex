\begin{lstlisting}[language=JAVA, caption=Algoritmo Base del Método de Cuadrantes, label=lst:codigo11,escapechar=|]
funcion Inicio()
 {
	RutaSolucion{Puntos[n]}
	ArregloCuadrantes{Cuadrantes[Puntos[n]]}
	Limite = n
	Direccion = derecha
	SubcuadranteOrigen = -1
	CuadranteOrigen = -1
	CuadranteDestino = -1
	Comprobar()
}
funcion Comprobar()
{
	 While (ArregloCuadrantes.length > 0)
	 {
		CuadrantesActual = ArregloCuadrantes[0]
		l = 0
		While CuadrantesActual.length < l {
			PuntosActual = CuadrantesActual[l]
			If(PuntosActual.length < Limite)
			{
				Resolver()
			} 
			else 
			{
				Dividir()  
			}
			l++
		} 
		ArregloCuadrantes.remover[0]
	}
	Terminar()
}	
funcion Dividir()
{
	NuevosCuadrantes{Cuadrantes[Puntos[n]],Cuadrantes[Puntos[n]],Cuadrantes[Puntos[n]],Cuadrantes[Puntos[n]]} = Dividir PuntosActual en 4
	Se reordena el orden de los elementos de NuevosCuadrantes de acuerdo al automata de cuadrantes de la figura \ref {fig:automata.png}
		  -Para ello se calculan 2 ejes (X y Y) a partir de los valores que forman los puntos de ese cuadrante, de ahi se establece el centro que es igual a los valores divididos en 2
		  - A partir de ahi se establece el orden de cuadrantes que va a recorrer para completar la ruta basandose en
				-SubcuadranteOrigen: El cuadrante donde se encuentra
				-En caso de que el cuadrante pertenezca a uno mas grande, el cuadrante al que planea dirigirse el cuadrante padre
					-En caso de no ser menor a 0 significa que planea regresar a si mismo, por tanto puede viajar a direccion, izquierda,derecha,arriba o abajo (por preferencia se eligio izquierda), en el automata de la figura |\ref {fig:automata.png}| se interpretara como estado |$\lambda$|                    
				-En caso de que el cuadrante pertenezca a uno mas grande, el numero de cuadrante padre al que pertenece
					-En caso de no ser menor a 0 significa que planea regresar a si mismo, por tanto puede viajar a dirección, izquierda,derecha,arriba o abajo (por preferencia se eligio izquierda), en el automata de la figura |\ref {fig:automata.png}| se interpretara como estado |$\lambda$|       
	Agregar NuevosCuadrantes a ArregloCuadrantes reemplazando PuntosActual en el mismo lugar donde estaban
	Comprobar()	
}
funcion Resolver()
{
	PuntosActual = Reordenar orden de los elementos de PuntosActual usando metodo de Fuerza Bruta probando todas las combinaciones posibles y obtener el mejor resultado
	agregar PuntosActual a RutaSolucion
	Comprobar()
}
function Terminar()
{
	Mostrar resultados de RutaSolucion
}	
\end{lstlisting}  